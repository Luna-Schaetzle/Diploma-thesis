%% Template basiert auf der Vorlage der Uni Graz für VWA: https://latex.tugraz.at/vorlagen/allgemein

%% Versionen:
%% V1: 9. Augugst 2021 (GreiA)



%% %%%%%%%%%%%%%%%%%%%%%%%%%%%%%%%%%%%%%%%%%%%%%%%%%%%%%%%%%%%%%%%%%%%
%% Allgemeine Festlegungen für die Darstellung des Gesamtdokumentes;
%% %%%%%%%%%%%%%%%%%%%%%%%%%%%%%%%%%%%%%%%%%%%%%%%%%%%%%%%%%%%%%%%%%%%
\newcommand{\mypapersize}{A4}
%% e.g., "A4", "letter", "legal", "executive", ...
%% The size of the paper of the resulting PDF file.

\newcommand{\mylaterality}{twoside}
%% "oneside" or "twoside"
%% Either you are creating a document which is printed on both, left pages
%% and right pages (twoside) or you create a document which is printed
%% on right pages only (oneside).

\newcommand{\mydraft}{false}
%% "true" or "false"
%% Use draft mode? If true, included graphics are replaced by empty
%% rectangles (of same size) and overfull boxes (in margin space) are
%% marked with black box (-> easy to spot!)

\newcommand{\myparskip}{half}
%% e.g., "no", "full", "half", ...
%% How to separate paragraphs: indention ("no") or spacing ("half",
%% "full", ...).

\newcommand{\myBCOR}{0mm}
%% Inner binding correction. This value depends on the method which is
%% being used to bind your printed result. Some techniques do not
%% require a binding correction at all ("0mm"), other require for
%% example "5mm". Refer to KOMA script documentation for a detailed
%% explanation what a binding correction is and how to measure it.

\newcommand{\myfontsize}{12pt}
%% e.g., 10pt, 11pt, 12pt
%% The font size of the main text in pt (points).

\newcommand{\mylinespread}{1.0}
%% e.g., 1.0, 1.5, 2.0
%% Line spacing in %/100. For example 1.5 means 150% of the usual line
%% spacing. Please use with caution: 100% ("1.0") is fine because the
%% font was designed for it.

\newcommand{\mylanguage}{american,ngerman}
%% "english,ngerman", "ngerman,english", ...
%% NOTE: The *last* language is the active one!
%% See babel documentation for further details.

%% BibLaTeX-settings: (see biblatex reference for further description)
\newcommand{\mybiblatexstyle}{authoryear}
%% e.g., "alphabetic", "authoryear", ...
%% The biblatex style which is being used for referencing. See
%% biblatex documentation for further details and more values.
%%
%% CAUTION: if you change the style, please check for (in)compatible
%%          "biblatex" package options in the file
%%          "template/preamble.tex"! For example: "alphabetic" does
%%          not have an option "dashed=..." and causes an error if it
%%          does not get removed from the list of options.

\newcommand{\mybiblatexdashed}{false}  %% "true" or "false"
%% If true: replace recurring reference authors with a dash.

\newcommand{\mybiblatexbackref}{true}  %% "true" or "false"
%% If true: create backward links from reference to citations.

\newcommand{\mybiblatexfile}{references-biblatex.bib}
%% Name of the biblatex file that holds the references.

\newcommand{\mydispositioncolor}{30,103,182}
%% e.g., "30,103,182" (blue/turquois), "0,0,0" (black), ...
%% Color of the headings and so forth in RGB (red,green,blue) values.
%% NOTE: if you are using "0,0,0" for black, printers might still
%%       recognize pages as color pages. In case this is a problem
%%       (paying for color print-outs vs. paying for b/w-printouts)
%%       please edit file "template/preamble.tex" and change
%%       "\definecolor{DispositionColor}{RGB}{\mydispositioncolor}"
%%       to "\definecolor{DispositionColor}{gray}{0}" and thus
%%       overwriting the value of \mydispositioncolor above.

\newcommand{\mycolorlinks}{true}  %% "true" or "false"
%% Enables or disables colored links (hyperref package).



\newcommand{\mytitlepage}{template/title_thesis_htlinn} %Titelseite

\newcommand{\mytodonotesoptions}{}
%% e.g., "" (empty), "disable", ...
%% Options for the todonotes-package. If "disable", all todonotes will
%% be hidden (including listoftodos).

%% Load main settings for document preamble:
\documentclass[%
fontsize=\myfontsize,%% size of the main text
paper=\mypapersize,  %% paper format
parskip=\myparskip,  %% vertical space between paragraphs (instead of indenting first par-line)
DIV=calc,            %% calculates a good DIV value for type area; 66 characters/line is great
headinclude=true,    %% is header part of margin space or part of page content?
footinclude=false,   %% is footer part of margin space or part of page content?
open=right,          %% "right" or "left": start new chapter on right or left page
appendixprefix=true, %% adds appendix prefix; only for book-classes with \backmatter
bibliography=totoc,  %% adds the bibliography to table of contents (without number)
draft=\mydraft,      %% if true: included graphics are omitted and black boxes
                     %%          mark overfull boxes in margin space
BCOR=\myBCOR,        %% binding correction (depends on how you bind
                     %% the resulting printout.
\mylaterality        %% oneside: document is not printed on left and right sides, only right side
                     %% twoside: document is printed on left and right sides
]{scrbook}  %% article class of KOMA: "scrartcl", "scrreprt", or "scrbook".
            %% CAUTION: If documentclass will be changed, *many* other things
            %%          change as well like heading structure, ...

\usepackage{float}
\usepackage{ifthen}
\usepackage[T1]{fontenc}
\usepackage{lmodern}
\usepackage[utf8]{inputenc} %% UTF8 as input characters
\usepackage{textcomp}
\usepackage[\mylanguage]{babel}  %% used languages; default language is *last* language of options
\usepackage{scrlayer-scrpage} %%  advanced page style using KOMA


\ifthenelse{\boolean{\mydraft}}{   %% the \mydraft switches between
                                   %% showing rectangles instead of graphics
  \usepackage[pdftex,draft]{graphicx}
}
{
  \usepackage[pdftex]{graphicx}
}

\usepackage{pifont}
%% pre-define ifthen-boolean variables:
\newboolean{myaddcolophon}
\newboolean{myaddlistoftodos}
\newboolean{english_affidavit}
\usepackage{xspace}
\usepackage[usenames,dvipsnames]{xcolor}
\definecolor{DispositionColor}{RGB}{\mydispositioncolor} %% used for links and so forth in screen-version

\usepackage[normalem]{ulem}
\usepackage{framed}
\usepackage{eso-pic}
\usepackage{enumitem}
\usepackage[\mytodonotesoptions]{todonotes}  %% option "disable" removes all todonotes output from 
\usepackage{units}
\usepackage{listings}
%% DO NOT REMOVE THIS LINE!

\setboolean{myaddcolophon}{true}  %% "true" or "false"
%% If set to "true": a colophon (with notes about this document
%% template, LaTeX, ...) is added after the title page.
%% Please do not set to "false" without a good reason. The colophon
%% helps your readers to get in touch with LaTeX and to find this template.

\setboolean{myaddlistoftodos}{false}  %% "true" or "false"
%% If set to "true": the current list of open todos is added after the
%% table of contents. If \mytodonotesoptions is set to "disable", no
%% list of todos is added, independent of this setting here.

\setboolean{english_affidavit}{true}  %% "true" or "false"
%% If set to "true": the language of the statutory declaration text is set to
%% English, otherwise it is in German.




%% ========================================================================
%% Document metadata: DIESE WERTE BITTE ANPASSEN, wie werden dann automatisch auf der
%% Titelseite angezeigt
%% ========================================================================

%\newcommand{\myprojectpartner}{XXX} 
\newcommand{\mytitle}{AI in the Industry and Education Environment} %% Titel der Arbeit
\newcommand{\mysubtitle}{SUBTITITLE}
\newcommand{\myinstitute}{Abteilung für Wirtschaftsingenieure/Betriebsinformatik} 
\newcommand{\mysubmissionyear}{2025} %% Einreich - Jahr
\newcommand{\mysubmissionmonth}{April} %% Monat der Einreichung
\newcommand{\myauthor}{Gabriel Mrkonja\\Florian Prandstetter\\Luna Schätzle}  %% Autoren. Bitte mit \\ Trennen wenn mehrere
\newcommand{\mysupervisor}{Greinöcker\\Egger}  %%Betreuer. Bitte mit \\ Trennen wenn mehrere
%%\newcommand{\myprojectpartner}{Vollständige Bezeichnung der Firma}  %% Partnerfirma

\newcommand{\mysubject}{SUBJECT}  %% also used for PDF metadata (hyperref)
\newcommand{\mykeywords}{KEYWORDS}  %% also used for PDF metadata (hyperref)


%% header settings
\usepackage{lastpage}

\ohead{\headmark }
\ihead*{\includegraphics[width=3cm]{figures/htl-logo}}

\ifoot{\thepage}  %Will man Anzahl Seiten: /\pageref{LastPage}
\ofoot{\myauthor}


%% ========================================================================
%%%% MISC command definitions
%% ========================================================================
\input{template/mycommands}

%% ========================================================================
%%%% Typographic settings
%% ========================================================================
\input{template/typographic_settings}
%%Java Code im Stil von Eclipse

\definecolor{javared}{rgb}{0.6,0,0} % for strings
\definecolor{javagreen}{rgb}{0.25,0.5,0.35} % comments
\definecolor{javapurple}{rgb}{0.5,0,0.35} % keywords
\definecolor{javadocblue}{rgb}{0.25,0.35,0.75} % javadoc
 
\lstdefinestyle{Java}{language=Java,
basicstyle=\ttfamily,
keywordstyle=\color{javapurple}\bfseries,
stringstyle=\color{javared},
commentstyle=\color{javagreen},
morecomment=[s][\color{javadocblue}]{/**}{*/},
numbers=left,
numberstyle=\tiny\color{black},
stepnumber=1,
numbersep=10pt,
tabsize=4,
showspaces=false,
showstringspaces=false}


%%Python Code im Stil von Eclipse

\definecolor{red}{rgb}{1,0,0} % keywords

\lstdefinestyle{Python}{language=Python,
	basicstyle=\ttfamily,
	keywordstyle=\color{javapurple}\bfseries,
	stringstyle=\color{javared},
	commentstyle=\color{javagreen},
	morecomment=[s][\color{javadocblue}]{/**}{*/},
	numbers=left,
	numberstyle=\tiny\color{black},
	stepnumber=1,
	numbersep=10pt,
	tabsize=4,
	showspaces=false,
	showstringspaces=false}

%% ========================================================================
%%%% MISC usepackages
%% ========================================================================

%% ... it's OK to put here your own usepackage commands ...




%% ========================================================================
%%%% MISC self-defined commands and settings
%% ========================================================================

%% ... it's OK to put here your own newcommand/newenvironment-definitions ...





\hyphenation{ex-am-ple hy-phen-ate}  %% in order to use German umlauts
%% here (Ver-\"of-fent-li-chung), you have to check for
%% activated \usepackage[T1]{fontenc} in the preamble

%% override default language of babel: (be sure to know, what you're
%% doing here)
\selectlanguage{american}
%\selectlanguage{ngerman}

%% ========================================================================
%% bibtex für die Literaturverwaltung: Hier wird der Zitier-Stil festgelegt
%% ========================================================================
\usepackage{natbib} 
\bibliographystyle{agsm} 





\input{template/pdf_settings}  %% should be *last* definitions in preamble!

%% ========================================================================
%%%% begin{document}
%% ========================================================================

\begin{document} 
\frontmatter                    %% KOMA: roman page numbers and such; only available in scrbook

%% \input{colophon}                %% defines information about editor, LaTeX, font, ...

%% Choose your desired title page:
\input{\mytitlepage}            %% include title page


\begin{center}
	
{\LARGE SPERRVERMERK}

\vfill
Auf Wunsch der Firma \\ 
\vfill
{\large \myprojectpartner } \\
\vfill
 ist die vorliegende Diplomarbeit \\

für die Dauer von drei / fünf / sieben Jahren \\
für die öffentliche Nutzung zu sperren. \\
Veröffentlichung, Vervielfältigung und Einsichtnahme sind ohne \\ 
ausdrückliche Genehmigung der Firma *** und der Verfasser \\
bis zum TT.MM.JJJJ nicht gestattet.
\vfill
Innsbruck, TT.MM.JJJJ
\vfill
Verfasser:
\vfill\vfill
Vor- und Zuname	 \hspace{3cm}	Unterschrift
\vfill\vfill
Vor- und Zuname	 \hspace{3cm}	Unterschrift
\vfill\vfill
Firma:  \hspace{3cm} Firmenstempel


\end{center} % Wenn kein Sperrvermerk gemacht werden soll, dann diesen Import einfach auskommentieren

%%\input{template/declaration_TU_Graz}  %% Statutory Declaration
% \input{thanks}                %% this is a suggestion: you have to create this file on demand
% \input{foreword}              %% this is a suggestion: you have to create this file on demand


%% include the abstract without chapter number but include it on table of contents:
%%\let\cleardoublepage\clearpage
\phantomsection
\addcontentsline{toc}{chapter}{Abstract}

%%%% Time-stamp: <2013-02-25 10:31:01 vk>


\chapter*{Kurzfassung /Abstract }
\label{cha:abstract}

Eine Kurzfassung ist in deutscher sowie ein Abstract in englischer Sprache mit je maximal einer A4-Seite zu erstellen. Die Beschreibung sollte wesentliche Aspekte des Projektes in technischer Hinsicht beschreiben. Die Zielgruppe der Kurzbeschreibung sind auch Nicht-Techniker! Viele Leser lesen oft nur diese Seite. \\ \\

Beispiel für ein Abstract (DE und EN) \\ \\
Die vorliegende Diplomarbeit beschäftigt sich mit verschiedenen Fragen des Lernens Erwachsener – mit dem Ziel, Lernkulturen zu beschreiben, die die Umsetzung des Konzeptes des Lebensbegleitenden Lernens (LBL) unterstützen. Die Lernfähigkeit Erwachsener und die unterschiedlichen Motive, die Erwachsene zum Lernen veranlassen, bilden den Ausgangspunkt dieser Arbeit. Die anschließende Auseinandersetzung mit Selbstgesteuertem Lernen, sowie den daraus resultierenden neuen Rollenzuschreibungen und Aufgaben, die sich bei dieser Form des Lernens für Lernende, Lehrende und Institutionen der Erwachsenenbildung ergeben, soll eine erste Möglichkeit aufzeigen, die zur Umsetzung dieses Konzeptes des LBL beiträgt. Darüber hinaus wird im Zusammenhang mit selbstgesteuerten Lernprozessen Erwachsener die Rolle der Informations- und Kommunikationstechnologien im Rahmen des LBL näher erläutert, denn die Eröffnung neuer Wege zur orts- und zeitunabhängiger Kommunikation und Kooperation der Lernenden untereinander sowie zwischen Lernenden und Lernberatern gewinnt immer mehr an Bedeutung. Abschließend wird das Thema der Sichtbarmachung, Bewertung und Anerkennung des informellen und nicht-formalen Lernens aufgegriffen und deren Beitrag zum LBL erörtert. Diese Arbeit soll einerseits einen Beitrag zur besseren Verbreitung der verschiedenen Lernkulturen leisten und andererseits einen Reflexionsprozess bei Erwachsenen, die sich lebensbegleitend weiterbilden, in Gang setzen und sie somit dabei unterstützen, eine für sie geeignete Lernkultur zu finden. \\ \\


This thesis deals with the various questions concerning learning for adults – with the aim to describe learning cultures which support the concept of live-long learning (LLL). The learning ability of adults and the various motives which lead to adults learning are the starting point of this thesis. The following analysis on self-directed learning as well as the resulting new attribution of roles and tasks which arise for learners, trainers and institutions in adult education, shall demonstrate first possibilities to contribute to the implementation of the concept of LLL. In addition, the role of information and communication technologies in the framework of LLL will be closer described in context of self-directed learning processes of adults as the opening of new forms of communication and co-operation independent of location and time between learners as well as between learners and tutors gains more importance. Finally the topic of visualisation, validation and recognition of informal and non-formal learning and their contribution to LLL is discussed. \\

Gliederung des Abstract in \textbf{Thema}, \textbf{Ausgangspunk}, \textbf{Kurzbeschreibung}, \textbf{Zielsetzung}.  

\subparagraph{Projektergebnis}

Allgemeine Beschreibung, was vom Projektziel umgesetzt wurde, in einigen kurzen Sätzen. Optional Hinweise auf Erweiterungen. Gut machen sich in diesem Kapitel auch Bilder vom Gerät (HW) bzw. Screenshots (SW).
Liste aller im Pflichtenheft aufgeführten Anforderungen, die nur teilweise oder gar nicht umgesetzt wurden (mit Begründungen).              %% Abstract
\chapter*{Erklärung der Eigenständigkeit der Arbeit}
\label{cha:affirmation}


\textbf{EIDESSTATTLICHE ERKLÄRUNG }
\vspace{1cm}

Ich erkläre an Eides statt, dass ich die vorliegende Arbeit selbständig und ohne fremde Hilfe verfasst, andere als die angegebenen Quellen und Hilfsmittel nicht benutzt und die den benutzten Quellen wörtlich und inhaltlich entnommenen Stellen als solche erkenntlich gemacht habe. Meine Arbeit darf öffentlich zugänglich gemacht werden, wenn kein Sperrvermerk vorliegt.

\vfill\vfill



Ort, Datum  \hspace{5cm}Verfasser  1     
\vfill    
Ort, Datum  \hspace{5cm}Verfasser  1                                                  
 %%EIDESSTATTLICHE ERKLÄRUNG  

\tableofcontents                %% this produces the table of contents - you might have guessed :-)



%% if myaddlistoftodos is set to "true", the current list of open todos is added:
\ifthenelse{\boolean{myaddlistoftodos}}{
  \newpage\listoftodos          %% handy if you are using todonotes with \todo{}
}{}                             %% with todonotes-package option "disable" you can get rid of any todo in the output

\mainmatter                     %% KOMA: marks main part using arabic page numbers and such; only available in scrbook


%% HIER DIE EIGENEN KAPITEL EINFÜGEN
\part{Introduction}

\chapter{Einleitung}

In der Einleitung wird erklärt, wieso man sich für dieses Thema entschieden hat. (Zielsetzung und Aufgabenstellung des Gesamtprojekts, fachliches und wirtschaftliches Umfeld)

\section{Vertiefende Aufgabenstellung}

\subsection{Schüler*innen Name 1}

\subsection{Schüler*innen Name 2}

\section{Dokumentation der Arbeit}

Es werden die Projektergebnisse dokumentiert

\begin{itemize}
	\item Grundkonzept
	\item Theoretische Grundlagen
	\item Praktische Umsetzung
	\item Lösungsweg
	\item Alternativer Lösungsweg
	\item Ergebnisse inkl. Interpretation
\end{itemize}

Weitere Anregungen:

\begin{itemize}
	\item Fertigungsunterlagen
	\item Testfälle (Messergebnisse…)
	\item Benutzerdokumentation
	\item Verwendete Technologien und Entwicklungswerkzeuge
\end{itemize} %Delet this later/ its just a Reminder
\chapter{Introduction: AI in the Industry and Education Environment}
\label{chap:introduction}

% write something about the current Rise of AI and the difficulties of the implementation in the industry and education environment
% Then split the students and thier main field of study in the Diploma thesis


% Maby write something about the whay we made the documentations




  

\part{Hardware}
%Hardware
\chapter{Raspberry PI}
\label{chap:Raspberry PI}
\chapter{Server}
\label{chap:Server}

% Introduction why we need a Server
% write about the Server more on a Hardware level
% OS Level is in the next Chapter
% write about the Server and what software is running on it 


\part{Implementation} %search for a better name Maby: Software or Theoretical Part or Software background
%Software
\chapter{Used Technologies} % maby change the name
\label{chap:used_technologies}

\section{Introduction}

%add later

\section{Visual Studio Code}

\section{Vue.js}

\section{firebase}

\section{Github} % maby not needed
    



\chapter{Operating Systems used}
\label{chap:Operating_Systems_used}

% make some introduction
% add the os that were evaluated for the project
% write the outcome of the evaluation
% write the reason why the os was chosen
% write the reason why the os was not chosen
% Maby about Raspberry Pi OS (Maby in Gabis Part)





% Maby merch with used technologies

\chapter{Used Programming Languages}
\label{chap:used_programming_languages}



\chapter{API and Libraries}
\label{chap:API_Libaryies}



\part{Implementation of Artificial Intelligence}
\chapter{Introduction to the used AI Models}
\label{cha:Introduction_to_the_used_AI_Models}

For the Diploma thesis, there are many different AI models that are in use. There are different Types of AI models, such as:
\begin{itemize}
    \item LLMs (Large Language Models)
    \item Defusion Models (Models that are used to create images)
    \item Object Detection Models (Models that are used to detect objects in images)
    \item Face Recognition Models (Models that are used to recognize faces in images)
\end{itemize}

In the following chapters, the different Types and the used models will be explained in more detail.

\section{LLMs}

-- Insert explanation of LLMs here --

source: https://www.ibm.com/topics/large-language-models

\section{Used LLMs}





\chapter{Ollama}
\label{cha:Ollama}

% =========================================================
% This is a Deep Dive into the Ollama Programm / Service
% Maby delet later
% =========================================================


\chapter{hosted Flask Service}
\label{cha:hosted_flask_service}

% Hier beschreiben wir wie wir unseren Flask Service gehostet bzw. Geschreiben haben und im Server implementiert haben


% Weitere Infos wie wir das Projekt umgesetzt haben bzw. wie wir die AI Modelle implementiert haben
% Flo -> Extension for Visual Studio Code

\part{Evaluations} %search for a better name
%Economics
\chapter{Open source evaluation on Economics}
\label{cha:Open_source_evaluation_Economics}

\section{Introduction}

\subsection{What is Open Source?}

Open Source represents a collaborative and transparent approach to software development and distribution, 
where the source code is made publicly accessible. This philosophy empowers users not only to utilize the software but also to modify, 
improve, and redistribute it freely. By fostering an environment of openness and collaboration, 
Open Source drives innovation and democratizes access to technology.

Linus Torvalds, the creator of the Linux operating system, encapsulated this spirit of freedom and collaboration with his famous remark:

\begin{quote}
    \textit{“Software is like sex: it's better when it's free.”}
    \author{Linus Torvalds}
\end{quote}

\cite{Linus_Torvalds_quote_open_source}

This statement highlights the fundamental ethos of Open Source—the belief that open access and shared knowledge result in better, more impactful solutions.


The development process for Open Source software is often a collective effort, 
with contributions from diverse communities of developers, users, and organizations. 
These collaborative efforts enhance the software's functionality, security, and usability, 
resulting in products that are robust and adaptable. Prominent examples include the Linux operating system, 
the Apache web server, and the Firefox web browser, all of which have significantly influenced technological innovation and market dynamics.

\subsection{Advantages of Open Source}

Open Source software offers a wide range of benefits, making it a cornerstone of modern technology:

\begin{itemize}
    \item \textbf{Cost Efficiency:} Open Source software is typically free of charge, helping organizations and individuals save on licensing and maintenance costs.
    \item \textbf{Flexibility:} Users can access the source code, enabling them to tailor the software to their specific needs and requirements.
    \item \textbf{Security:} The open nature of the source code allows for peer review, ensuring vulnerabilities are identified and addressed promptly.
    \item \textbf{Community Support:} Open Source projects often benefit from vibrant developer communities, providing updates, patches, and user assistance.
    \item \textbf{Innovation:} The collaborative ecosystem of Open Source encourages creativity, leading to groundbreaking solutions and advancements.
    \item \textbf{Compatibility:} Many Open Source projects are designed to integrate seamlessly with existing systems, reducing technical barriers.
    \item \textbf{Transparency:} Open access to the source code ensures that users can understand and verify how the software operates.
    \item \textbf{Freedom:} Users are granted the liberty to use, modify, and share the software without restrictive licensing agreements.
\end{itemize}

\subsection{Why Do People Use Open Source?}

The adoption of Open Source software is motivated by several compelling factors:

\begin{itemize}
    \item \textbf{Control:} Users gain full control over the software, enabling customization and optimization for specific use cases.
    \item \textbf{Cost Savings:} The absence of licensing fees significantly reduces expenses, making Open Source particularly attractive for startups and educational institutions.
    \item \textbf{Security:} Transparency in the source code allows for thorough auditing, enhancing trust and reliability.
    \item \textbf{Community:} The collaborative spirit of Open Source connects users with knowledgeable communities that share resources and support.
    \item \textbf{Stability:} Many Open Source projects offer long-term support and regular updates, ensuring reliability over time.
    \item \textbf{Skill Development:} Learning and using Open Source tools are valuable in educational and professional contexts, equipping individuals with in-demand skills.
\end{itemize}


\subsection{Chapter Overview}

This chapter introduces the concept of Open Source and highlights its significance in the modern economy. 
Key aspects such as the advantages and disadvantages of Open Source, as well as the challenges associated with its adoption and creation, are discussed. 
Additionally, the chapter explores revenue models within the Open Source ecosystem and its role in economic systems. 
Finally, the chapter concludes by presenting the Open Source tools utilized in this project, alongside a reflection on the experiences gained through their application \cite{opensource_what_is}.

%\bibliographystyle{plain}
%\bibliography{references}


\section{What is and isn’t Open Source?}

\subsection{Definition and Guiding Principles}

Open Source, as defined by the Open Source Initiative (OSI), is a development approach that prioritizes accessibility and transparency of software source code. It allows users to view, modify, and distribute the code freely, fostering collaboration and innovation. 

The OSI outlines several key principles that define Open Source software:

\begin{itemize}
    \item \textbf{Free Redistribution:} The software can be freely shared and distributed without restrictions.
    \item \textbf{Source Code Access:} Users must have access to the source code to study, modify, and improve the software.
    \item \textbf{Modification and Sharing:} Users are allowed to create and share modified versions, as long as they follow the license terms.
    \item \textbf{No Discrimination:} The software must be available for everyone, regardless of individual characteristics or professional field.
    \item \textbf{Neutrality and Compatibility:} The license must not favor specific technologies or restrict the use of other software.
\end{itemize}

These principles ensure that Open Source remains a transparent, inclusive, and adaptable approach to software development, enabling innovation and collaboration across industries and communities.

\cite{Open_Source_Initiative_OS_definition}

\subsection{Misconceptions About Open Source}

Open Source is often misunderstood and confused with other software distribution models, which can lead to misconceptions about its nature, functionality, and benefits. 
It is crucial to distinguish Open Source from other types of software:

\begin{itemize}
    \item \textbf{Open Source:} Software that is freely accessible, modifiable, and redistributable under an Open Source license, adhering to principles such as transparency and collaboration.
    \item \textbf{Freeware:} Software available at no cost but typically without access to the source code, meaning users cannot modify or redistribute it.
    \item \textbf{Proprietary Software:} Software owned and controlled by a single entity, restricting access to the source code and preventing users from making modifications or redistributions.
    \item \textbf{Commercial Software:} Software sold for profit, which may be either Open Source or proprietary, depending on the licensing terms.
\end{itemize}

Understanding these distinctions helps users make informed choices about software selection and ensures their expectations align with the capabilities and freedoms provided by the chosen software.

To verify whether a software is truly Open Source, it is essential to examine the license agreement and confirm the availability of the source code. 
Software with an OSI-approved license is a reliable indicator that it adheres to Open Source principles, providing transparency, freedom, and collaboration opportunities.

One common misconception about Open Source software arises from the phrase "free as in freedom" versus "free as in free beer."  
While "free as in freedom" emphasizes the liberty to access, modify, and share the software, "free as in free beer" simply denotes that the software is free of cost. 
Although Open Source software is often available without charge, its true value lies in the freedoms it grants to users, developers, and organizations. 
This distinction highlights the broader significance of Open Source as a philosophy, not just a pricing model.

\cite{forbes_misconceptions_open_source_2024}


\section{The Role of Open Source in Economics}

Cost efficiency, innovation, and collaboration are key factors that have positioned Open Source as a cornerstone of modern economic systems. M
any industries and organizations utilize Open Source software to reduce costs, increase flexibility, 
and promote creativity, thereby driving economic growth and sustainability.

\subsection{Driving Innovation and Shaping Market Dynamics}

Open Source software fosters a culture of experimentation, creativity, and knowledge sharing, 
leading to the rapid development of new technologies and solutions. By granting users access to modify and redistribute the source code, 
Open Source encourages collaboration and innovation, 
enabling individuals and organizations to build upon existing software to create new products and services.

A distinctive strength of Open Source is its inclusivity—anyone, regardless of their affiliation with a company,
can contribute to its development. 
This openness lowers barriers to entry for innovation and allows passionate individuals to make meaningful contributions.

Companies also play a significant role in advancing Open Source projects. 
With greater resources and structured teams, organizations can contribute in a more organized and impactful manner, 
accelerating development and enhancing software quality.

The collaborative nature of Open Source facilitates cross-industry partnerships, 
allowing organizations from diverse sectors to share knowledge, resources, and best practices. 
This cross-pollination of ideas not only enhances software development but also fosters innovation across industries, 
ultimately shaping market dynamics and driving economic progress.

The study \cite{opensource_hendrickson2012economic} by Mike Hendrickson, Roger Magoulas, 
and Tim O'Reilly underscores that Open Source is not only a catalyst for small business growth but also a driver of future success for many startups today. 
By providing cost-effective and flexible solutions,
Open Source enables small and medium-sized enterprises to strengthen their online presence and enhance their economic performance.


\subsection{Supporting Startups and small Enterprises}

The impact of Open Source on startups and small enterprises is both profound and transformative. 
For these businesses, Open Source software provides a highly cost-effective alternative to proprietary solutions, 
granting access to advanced tools and technologies without the financial burden of high licensing fees typically associated with commercial software. 
This affordability allows startups and small enterprises to allocate their limited resources more strategically,
fostering innovation and growth while maintaining financial flexibility.

\cite{studiolabs_open_source_startups_2024}

\subsection{Enabling Cross-Industry Collaboration and Open Innovation}

% eventuell löschen

\section{Advantages and Disadvantages of Open Source}
\subsection{Advantages}

Open Source software offers numerous advantages for users, developers, and businesses. It can vary from cost savings to increased innovation and flexibility for customization.
%continue


%Delete later
\begin{itemize}
    \item Cost savings.
    \item Flexibility for customization.
    \item Increased innovation due to open collaboration.
\end{itemize}

\subsection{Disadvantages}


%Delete later
\begin{itemize}
    \item Reliance on community support.
    \item Potential security vulnerabilities.
    \item Compatibility issues with other systems.
\end{itemize}

\section{Challenges of Using or Creating Open Source}

There are many challanges that come with using or creating Open Source software. These can range from technical to economic and social challenges. 
Understanding these challenges is crucial for successful Open Source adoption and development.

\subsection{Technical Challenges}


%Delete later
\begin{itemize}
    \item Maintaining quality and long-term compatibility.
    \item Managing security and privacy risks.
\end{itemize}

\subsection{Economic Challenges}

%Delete later
\begin{itemize}
    \item Monetization and sustainability concerns.
    \item Balancing free access with profitability.
\end{itemize}
\subsection{Social Challenges}

%Delete later
\begin{itemize}
    \item Effective community management and governance.
\end{itemize}
\subsection{Legal Issues}

%Delete later
\begin{itemize}
    \item Navigating complex licensing models (e.g., GPL, MIT).
\end{itemize}

\section{Revenue Models in Open Source}

Open Source projects can generate revenue through various business models, each with its own advantages and challenges.

%Delete later
\begin{itemize}
    \item Common business models:
    \begin{itemize}
        \item Freemium.
        \item Support and maintenance services.
        \item Dual licensing.
        \item Crowdfunding and donations.
    \end{itemize}
    \item Real-world examples of successful Open Source businesses (e.g., Linux, Red Hat, MySQL).
\end{itemize}

\section{Open Source in Key Industries}

%Delete later
\begin{itemize}
    \item The role of Open Source in transforming:
    \begin{itemize}
        \item Information Technology (e.g., operating systems, tools).
        \item Artificial Intelligence (e.g., TensorFlow, PyTorch).
        \item Education (e.g., Moodle, Jupyter Notebooks).
    \end{itemize}
    \item Governmental and policy support for Open Source adoption.
\end{itemize}

\section{Reflexion}

%Delete later
\begin{itemize}
    \item Answering the research question based on the above analysis.
    \item Evaluating the broader implications of Open Source for economic systems.
    \item Connecting Open Source's potential with sustainability and global development.
\end{itemize}

\section{Open Source in Practice: A Personal Experience}

%Delete later
\begin{itemize}
    \item Open Source tools and technologies used in the project:
    \begin{itemize}
        \item Python, Flask, Vue.js, Linux, wttr.in API, LLaMA API.
    \end{itemize}
    \item Challenges and solutions encountered:
    \begin{itemize}
        \item Technical hurdles.
        \item Why Open Source alternatives were chosen or rejected.
    \end{itemize}
    \item Comparison of Open Source and closed-source software used:
    \begin{itemize}
        \item Reasons for choosing closed-source alternatives where applicable.
    \end{itemize}
\end{itemize}

\section{Open Source in Our Project \& Licensing}
\subsection{Project}

%Delete later
\begin{itemize}
    \item Description of the project.
    \item How Open Source principles were applied.
    \item Benefits and challenges of Open Source in the project.
\end{itemize}
\subsection{License}

%Delete later
\begin{itemize}
    \item Choice of license and rationale.
    \item How the license aligns with the project’s goals.
    \item The license problems of the project.
    \item Future plans for the project’s development and licensing.
\end{itemize}

\section{Conclusion}

%Delete later
\begin{itemize}
    \item Summary of Open Source’s economic impact.
    \item Reflections on its potential to drive future innovation and growth.
    \item Final thoughts on your personal experience and insights gained.
\end{itemize}

 

\part{Conclusion}
\input{content/Conclusion}
\chapter{Proplemes that occured}
\label{chap:Problems_that_occured}
\chapter{Outlook}
\label{chap:Outlook}

%\chapter{Test}
\label{cha:test}

Hier ist ein kleiner Test für die Verwendung von \LaTeX.


asdfasdf
asdfasdfsa
defaultasd
fsad
fsaddf % Einfach auskommentieren für die tatsächliche Arbeit
%
\chapter{Latex-Beispiele}
\label{chap:bsp}

\section{Aulistungen}

\begin{itemize}
	\item \textit{Kursiv} Text 1
	\item \textbf{Fett}  
	\item \texttt{TT} 
	\end{itemize}
	
	Dasselbe durchnumeriert:
	
	\begin{enumerate}
		\item \textit{Kursiv} Text 1
		\item \textbf{Fett}  
		\item \texttt{TT} 
	\end{enumerate}


\section{Tabellen}

Eine Tabelle mit Testdaten:


\begin{table}[H]
	\begin{center}
		\begin{tabular}{lrrrrr}\hline\hline
			\multicolumn{1}{l}{\textbf{position}}&
			\multicolumn{1}{c}{\textbf{mean}}&
			\multicolumn{1}{c}{\textbf{median}}&
			\multicolumn{1}{c}{\textbf{sd}}&
			\multicolumn{1}{c}{\textbf{min}}&
			\multicolumn{1}{c}{\textbf{max}}
			\\ \hline
			\textbf{6}&$6.89$&$5.61$&$ 7.29$&$0.31$&$160.12$\\
			\textbf{9}&$5.35$&$4.39$&$ 4.94$&$0.18$&$ 76.40$\\
			\textbf{12}&$8.70$&$6.96$&$10.72$&$0.15$&$239.88$\\
			\textbf{13}&$9.01$&$7.54$&$ 7.60$&$0.15$&$138.86$\\
			\textbf{15}&$8.18$&$6.99$&$ 6.86$&$0.16$&$117.26$\\
			\textbf{16}&$5.26$&$4.42$&$ 4.99$&$0.08$&$110.21$\\
			\textbf{17}&$5.87$&$4.79$&$ 6.13$&$0.15$&$ 98.88$\\
			\textbf{36}&$8.21$&$6.72$&$ 7.58$&$1.36$&$122.35$\\
			\textbf{42}&$6.77$&$5.93$&$ 6.98$&$1.72$&$123.72$\\
			\textbf{43}&$6.27$&$5.53$&$ 3.21$&$0.57$&$ 35.69$\\
			\hline
		\end{tabular}
	\end{center}
	\caption{Eine Tabelle mit Testdaten} 
	\label{tabelle:test}
\end{table}

Sprachen wie z.B. \textbf{R} können Latex-Tabellen exportieren, sie müssen also nicht immer so aufwändig formatiert werden.			


\section{Abbildungen}

 \begin{figure}[H]
 	%\centering
 	\hspace*{-1.5cm}
 	\includegraphics[width=512pt,height=280pt]{figures/bsp.png}
 	\caption{Ein Beispiel für ein Bild}
 	\label{bild:beispiel}
 \end{figure}
 
 
\section{Quellcode}

Quellcode wird automatisch (mit der Möglichkeit die Sprache anzugeben) formatiert und in das Listings-Verzeichnis gegeben:

\subsection{Java-Code}

\begin{lstlisting}[style=Java, caption={Java-Beispiel}, captionpos=b]
int i = 1;
float f = 2;
System.out.printf("Int-Z %d Float-Z: 52f",i ,f );
\end{lstlisting} 


\subsection{Python-Code}
 
\begin{lstlisting}[style=Python, caption={Python-Beispiel}, captionpos=b]
#Hier ein kleines Beispiel in Python
lower = 0
upper = 10
for i in range(lower,upper):
print(i)
\end{lstlisting} 


\subsection{Lesen von Dateien}
 
Es kann auch direkt von Dateien gelesen werden:

\lstinputlisting[style=Java, label={java_bsp}, caption={Java-Beispiel von Datei}, captionpos=b]{sourcecode/First.java}
 
\section{Referenzen}
			
Beispiele für die Verwendung von Referenzen: 

\begin{itemize}
	\item Wie in Tabelle ~\ref{tabelle:test} ersichtlich... 
	\item Wir sind im Kapitel ~\ref{chap:bsp}
	\item In Zeile 2 im Listing ~\ref{java_bsp} 
\end{itemize}


\section{Zitate}


Hier das Zitat eines Buches: \cite{couper2001} Wird alles automatisch mit  bibtex erledigt. % Einfach auskommentieren für die tatsächliche Arbeit


\appendix                       %% closes main document, appendix follows until end; only available in book-classes

\addpart*{Appendix}             %% adding Appendix to tableofcontents



\listoftables
\listoffigures
\lstlistoflistings
\nocite{*} %Es werden auf nicht referenzierte Literaturstellen aufgelistet
\bibliography{references}

\end{document}
