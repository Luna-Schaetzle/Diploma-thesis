\chapter{Open source evaluation on Economics}
\label{cha:Open_source_evaluation_Economics}

\section{Introduction}

\subsection{What is Open Source?}

Open source is a way of developing and distributing software that allows users to make changes to the software and distribute it to others. The source code is made available to the public, so that anyone can use, modify, and distribute the software. 
Open source software is often developed by a community of developers who collaborate on the project. 
The open source model has been used to develop a wide range of software, including operating systems, web browsers, and office productivity tools.

Some famous examples of open source software include the Linux operating system, the Apache web server, and the Firefox web browser.

Some advantages of using open source software include:
\begin{itemize}
    \item Cost: Open source software is often available for free, which can save organizations money on software licensing fees.
    \item Flexibility: Because the source code is available, users can modify the software to meet their specific needs.
    \item Security: Open source software is often more secure than proprietary software, because the source code is available for review by security experts.
    \item Community support: Open source software is often supported by a community of developers who can help users troubleshoot problems and provide updates and patches.
    \item Innovation: The open source model encourages collaboration and innovation, which can lead to the development of new and better software.
    \item Compatibility: Open source software is often designed to be compatible with other software and systems, which can make it easier to integrate into existing systems.
    \item Transparency: Because the source code is available, users can see how the software works and verify that it is doing what it is supposed to do.
    \item Freedom: Open source software gives users the freedom to use, modify, and distribute the software as they see fit.
\end{itemize}



Source: https://en.wikipedia.org/wiki/Open_source